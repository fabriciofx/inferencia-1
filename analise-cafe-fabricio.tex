% Documento do tipo artigo, com fonte tamanho 11
\documentclass[11pt]{article}

% Configura o tamanho do papel e das margens
\usepackage[
    a4paper,
    top=20mm,
    right=20mm,
    bottom=10mm,
    left=20mm
]{geometry}

% Suporte a caracteres acentuados
\usepackage[utf8]{inputenc}

% Configura o idioma para português do Brasil, incluindo suporte a hifenização
\usepackage[brazil]{babel}

% Melhora a renderização da fonte Computer Modern em PDFs modernos
\usepackage{lmodern}

% Melhora a codificação da fonte
\usepackage[T1]{fontenc}

% Permite a inclusão de imagens / PDF como imagem
\usepackage{graphicx}

% Permite cores por nome
\usepackage{xcolor}

% Permite personalizar listas
\usepackage{enumitem}

% Configure o \includegraphics a não emitir a seguinte mensagem de warning:
% PDF inclusion: multiple pdfs with page group included in a single page
\pdfsuppresswarningpagegroup=1

% Começo do documento
\begin{document}

\begin{center}
UNIVERSIDADE FEDERAL DA PARAÍBA

CENTRO DE CIÊNCIAS EXATAS E DA NATUREZA

DEPARTAMENTO DE ESTATÍSTICA

DISCIPLINA: INFERÊNCIA I
\end{center}

\textbf{Docente:} Dra Tatiene Correia de Souza

\textbf{Discente:} Fabrício Barros Cabral -- 20230052696

\begin{center}
{\Large \textbf{Análise das Marcas de Café A e B para a CoffeMax}}
\end{center}

\begin{enumerate}
    \item \textbf{Qual fornecedor tem teor de cafeína mais adequado em média?}

    A marca A possui, em média, aproximadamente 95,18\% mg de cafeína por
    xícara, enquanto a marca B possui, em média, aproximadamente 95,30\% mg.
    Analisando estas médias com o nível do intervalo de confiança de 90\%, 95\%
    e 99\% verificou-se que a média não varia para a marca A ou para a marca B.
    Também verificou-se que, em média, \textbf{não há evidência estatística de
    diferença significativa do teor de cafeína entre as marcas}.

    \item \textbf{Qual fornecedor apresenta maior consistência (menor
    variabilidade)?}

    A marca A possui variância de aproximadamente 9,09 $(mg)^2$ por xícara de
    café, enquanto a marca B possui variância de aproximadamente 21,51 $(mg)^2$.
    Além disso, foi calculada a razão entre as variâncias do café da marca A e
    da marca B, para os níveis de intervalos de confiança de 90\%, 95\% e 99\%,
    que corroboraram que há uma diferença na variabilidade entre as marcas.
    Assim, concluímos que a marca B possui uma variabilidade maior que a marca A
    e, portanto, \textbf{a marca A apresenta maior consistência} quando
    comparada a marca B.

    \item \textbf{Qual fornecedor é mais aceito pelos consumidores?}

    Conforme a pesquisa de opinião com 120 consumidores para cada marca, em que
    a marca A foi aprovada por 78 consumidores e a marca B foi aprovada por 69
    consumidores, foram analisadas as proporções com intervalos de confiança de
    90\%, 95\% e 99\% e concluímos que \textbf{não há diferença estatística
    significativa entre as proporções de aprovação entre as duas marcas de
    café}.

    \item \textbf{Recomendação final}

    Considerando a análise das seguintes informações:

    \begin{enumerate}[label=\alph*)]
        \item que, em média, não há evidência estatística de diferença
        significativa do teor de cafeína entre as marcas A e B;
        \item que o fornecedor da marca A possui maior consistência;
        \item que não há diferença estatística significativa entre as proporções
        de aprovação entre os consumidores das duas marcas de café;
        \item que a empresa \textbf{CoffeeMax} quer lançar um \textit{blend}
        premium e, portanto, um café de melhor qualidade;
    \end{enumerate}

    \textbf{recomendamos à CoffeeMax a compra do café da marca A}, pois como a
    referida empresa visa lançar um café premium (de melhor qualidade), o café
    da marca A é mais previsível (consistente) do que o da marca B. Além disso,
    como a pesquisa de opinião com os consumidores não conseguiu
    (estatisticamente) distinguir qual o café é o mais aceito pelos
    consumidores, deve-se adotar um perfil conservador com relação à qualidade
    do café (menor variabilidade), pois o mesmo visa um público mais exigente
    com relação à marca escolhida.
\end{enumerate}
\end{document}
