% Documento do tipo artigo, com fonte tamanho 11
\documentclass[11pt]{article}

% Configura o tamanho do papel e das margens
\usepackage[
    a4paper,
    top=20mm,
    right=20mm,
    bottom=10mm,
    left=20mm
]{geometry}

% Suporte a caracteres acentuados
\usepackage[utf8]{inputenc}

% Configura o idioma para português do Brasil, incluindo suporte a hifenização
\usepackage[brazil]{babel}

% Melhora a renderização da fonte Computer Modern em PDFs modernos
\usepackage{lmodern}

% Melhora a codificação da fonte
\usepackage[T1]{fontenc}

% Permite a inclusão de imagens
\usepackage{graphicx}

% Permite cores por nome
\usepackage{xcolor}

% Configure o \includegraphics a não emitir a seguinte mensagem de warning:
% PDF inclusion: multiple pdfs with page group included in a single page
\pdfsuppresswarningpagegroup=1

% Começo do documento
\begin{document}

\begin{center}
UNIVERSIDADE FEDERAL DA PARAÍBA

CENTRO DE CIÊNCIAS EXATAS E DA NATUREZA

DEPARTAMENTO DE ESTATÍSTICA

DISCIPLINA: INFERÊNCIA I
\end{center}

\textbf{Docente:} Dra Tatiene Correia de Souza

\textbf{Discente:} Fabrício Barros Cabral -- 20230052696

\begin{center}
{\Large \textbf{Análise do Teor de Café}}
\end{center}

\begin{enumerate}
    \item Qual é a estimativa do teor médio de cafeína de cada fornecedor? Qual
    a margem de erro e a amplitude associadas às estimativas em diferentes
    níveis de confiança (90\%, 95\% e 99\%)? Interprete os resultados.

    A estimativa do teor médio de cafeína do fornecedor da marca A é de
    aproximadamente 95,18mg enquanto que para o fornecedor da marca B é de
    aproximadamente 95,30mg. As margens de erro e amplitudes associadas para
    o fornecedor da marca A são aproximadamente 0,6574 e 1,3148 para 90\% de
    confiança, 0,7892 e 1,7584 para 95\% de confiança e 1,0566 e 2,1131 para
    99\% de confiança. Já para o fornecedor da marca B, as margens de erro e
    amplitudes associadas, são aproximadamente 0,9243 e 1,8485 para 90\% de
    confiança, 1,1060 e 2,2119 para 95\% de confiança e 1,4685 e 2,9371 para
    99\% de confiança. Estes dados indicam que a média do teor de café da marca
    A é a mesma, independente se o valor do intervalo de confiança seja de 90\%,
    95\% ou 99\%. O mesmo vale para a média do teor de café da marca B.
    Entretanto, as margens de erro e as amplitudes aumentam (ficam mais largas)
    à medida que o intervalo de confiança aumenta em 90\%, 95\% e 99\%.

    \item Considerando a diferença entre as médias de cafeína das marcas A e B,
    qual é a estimativa dessa diferença? O intervalo sugere diferença
    signicativa no teor médio de cafeína entre as marcas em cada um dos níveis
    de confiança (90\%, 95\% e 99\%)?
    \item Seu chefe está preocupado com a regularidade do produto. Qual é a
    variabilidade do teor de cafeína de cada marca? Com base em intervalos de
    confiança de 95\% para as variâncias, qual fornecedor mostra maior
    consistência? Essa diferença pode impactar a percepção de qualidade pelo
    consumidor?
    \item Avaliando agora a relação entre as variabilidades, qual é a razão
    entre as variâncias das marcas A e B? Em diferentes níveis de conança (90\%,
    95\%, 99\%), o resultado sugere igualdade ou diferença na consistência entre
    os fornecedores?
    \item Pensando na pesquisa de opinião com consumidores, qual é a estimativa
    da proporção de aprovação de cada marca? Qual a margem de erro em 90\%, 95\%
    e 99\%? Qual marca se mostra mais aceita pelo público? Justique.
    \item Considerando a diferença entre as proporções de aprovação (A - B),
    qual é a estimativa dessa diferença? Em diferentes níveis de conança (90\%,
    95\%, 99\%), o intervalo inclui ou não o valor zero? O que isso indica sobre
    a aceitação das duas marcas no mercado?
\end{enumerate}



\end{document}
